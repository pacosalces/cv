%%%%%%%%%%%%%%%%%%%%%%%%%%%%%%%%%%%%%%%%%
% Medium Length Graduate Curriculum Vitae
% LaTeX Template
% Version 1.1 (9/12/12)
% Update to 1.1.1 (10/20/19), Francisco Salces--Carcoba
%
% This template has been downloaded from:
% http://www.LaTeXTemplates.com
%
% Original author:
% Rensselaer Polytechnic Institute (http://www.rpi.edu/dept/arc/training/latex/resumes/)
%
% Important note:
% This template requires the res.cls file to be in the same directory as the
% .tex file. The res.cls file provides the resume style used for structuring the
% document.
%
%%%%%%%%%%%%%%%%%%%%%%%%%%%%%%%%%%%%%%%%%

\documentclass[margin]{res} % Use the res.cls style, the font size can be changed to 11pt or 12pt here
\usepackage[sorting=ydnt, doi=false, maxbibnames=99, isbn=false, backend=bibtex]{biblatex}
\addbibresource{bibliography.bib}
\usepackage{hyperref}
\hypersetup{
    colorlinks = true,
    linkcolor = {blue},
}
\usepackage{helvet} % Default font is the helvetica postscript font
%\usepackage{newcent} % To change the default font to the new century schoolbook postscript font uncomment this line and comment the one above

\setlength{\textwidth}{5.1in} % Text width of the document
\begin{document}


\moveleft.5\hoffset\centerline{\large\bf Francisco Salces-C\'arcoba}
\moveleft.5\hoffset\centerline{\it Barish--Weiss Postdoctoral Scholar} 
\moveleft.5\hoffset\centerline{\vbox{\hrule width 0.5\resumewidth height 1pt}}\smallskip
\moveleft.5\hoffset\centerline{540 S Madison Ave Apt.2} 
\moveleft.5\hoffset\centerline{Pasadena, CA 91101}
\moveleft.5\hoffset\centerline{+1 (301) 222-3170}
\moveleft.5\hoffset\centerline{pacosalces@gmail.com}
\moveleft.5\hoffset\centerline{www.pacosalces.com}
%----------------------------------------------------------------------------------------
\begin{resume}

\section{EDUCATION}

{\bf PhD,} Physics \\
Universiy of Maryland College Park, College Park, MD, May 2020 \\
Title of dissertation: {\it Microscopy of elongated superfluids}

{\bf BSc,} Physics \\
Universidad Aut\'onoma de San Luis Potos\'i, San Luis Potos\'i, S.L.P., Mexico, 2013
 
\section{SKILLS AND EXPERTISE}

{\bf General:}
Experimental physics; gravitational wave detection, atomic, molecular, optical physics, quantum mechanics.

{\bf Hardware:}
{\underline{Optics};} passive and active optical design from 400 to 2300 nm, laser spectroscopy, adaptive optics, holography, coherent microscopy, optical cavities, and laser systems. {\underline{Vacuum};} high and ultra-high vacuum down to ${10^{-12} {\rm mbar}}$. {\underline {Electronics};} basic DC and AC applications into the radiofrecuency and microwave domain at $~10\,{\rm GHz}$. {\underline{Mechanics};} basic machining and assembly of components and structures.

{\bf Software:} 
{\underline{Numerical analysis \& hardware interfaces};} Python, Matlab, ImageJ, Labview, VHDL, bash, hpc. {\underline{CAD};} Solidworks, Zeemax, Eagle. {\underline{Other};} \LaTeX. 

Personal repositories; https://github.com/pacosalces

\section{RESEARCH TASKS}

{\sl Barish-Weiss Postdoctoral Fellow} \hfill 2020-present \\
{\bf Mariner: 40m prototype of a cryogenic gravitational wave detector}\\
{\it California Institute of Technology, Pasadena, CA}
\begin{itemize} \itemsep -2pt
\item $2.1\,{\rm \mu m}$ lasers for next-generation cryogenic gravitational wave interferometry.
\item $1.4\,{\rm \mu m}$ lasers for next-generation cryogenic gravitational wave detector arm-length stabilization.
\item Strain calibration for optimal gravitational wave detector parameter estimation.
\end{itemize} 

{\sl Graduate research assistant} \hfill 2017-2020 \\
{\bf Holographic microscopy of ultracold ${\bf ^{87} {\rm \bf Rb}}$}\\
{\it Joint Quantum Institute (University of Maryland and NIST), Gaithersburg, MD}
\begin{itemize} \itemsep -2pt
\item Off-axis digital holographic microscope for ${^{87} {\rm Rb}}$ gases.
\item Digital aberrations correction using holography and atom shot noise.
\end{itemize} 

{\sl Graduate research assistant} \hfill 2016-2018 \\
{\bf New apparatus for quantum degenerate Bose gases}\\
{\it Joint Quantum Institute (University of Maryland and NIST), Gaithersburg, MD}
\begin{itemize} \itemsep -2pt
\item Design and assembly of ultra-high vacuum manifolds.
\item Design of magnetic quadrupole based cold atom transport.
\item Design mounting structure and general layout for the apparatus.
\end{itemize}

{\sl Graduate research assistant} \hfill 2015-2017 \\
{\bf Thermodynamics of one-dimensional Bose gases} \\
{\it Joint Quantum Institute (University of Maryland and NIST), Gaithersburg, MD}
\begin{itemize} \itemsep -2pt
\item Design, build and characterize high-aspect ratio, cross optical dipole trap.
\item Design, build and characterize compound microscope objective (${{\rm NA} = 0.31}$).
\item Calibrate optimal absorption imaging of dilute ${\sim 1 \,{\rm \mu m ^{-1}}}$ linear gases.
\item Benchmark {\it in-situ} density distributions with numerically exact model.
\end{itemize} 

{\sl Graduate research assistant} \hfill 2014-2015 \\
{\bf Digital control loop for magnetic field stabilization}\\
{\it Joint Quantum Institute (University of Maryland and NIST), College Park, MD}
\begin{itemize} \itemsep -2pt
\item Design and simulate 20-bit, FPGA-based feedback system for electromagnets.
\end{itemize} 

{\sl Undergraduate research assistant} \hfill 2011-2013 \\
{\bf Passive thermal stabilization of optical cavities}\\
{\it Laboratorio de Atomos Frios, Instituto de Fisica UASLP, S.L.P, Mexico}
\begin{itemize} \itemsep -2pt
\item Design and assembly of composite material spacers for confocal cavity.
\item Measure thermally driven frequency drift of the linear cavity. 
\item Model thermal transient with 4th-order Runge--Kutta finite-element code.
\end{itemize}
 
{\sl Undergraduate research intern} \hfill Summer 2012 \\
{\bf Soft X-ray calorimetry from electron recapture}\\
{\it Oak Ridge National Lab, Oak Ridge, TN}
\begin{itemize} \itemsep -2pt
\item Operate ${{\rm keV}}$ molecular ion beam accelerator.
\item Operate high resolution X-ray cryogenic (${0.1 {\rm K}}$) calorimeter.
\end{itemize}

{\sl Undergraduate research assistant} \hfill 2009-2011 \\
{\bf Two-photon correlation functions $\bf g^{(2)}(\tau)$}\\
{\it Laboratorio de Atomos Frios, Instituto de Fisica UASLP, S.L.P, Mexico}
\begin{itemize} \itemsep -2pt
\item Automate measurements using two single photo detectors and oscilloscope.
\end{itemize}
 
\end{resume}

\section{PEER-REVIEWED PUBLICATIONS}

\nocite{*}

\printbibliography[heading=none, keyword={peer}]

\section{OTHER PUBLICATIONS}

\nocite{*}

\printbibliography[heading=none, keyword={npeer}]

\section{AFFILIATIONS}
 
\begin{itemize}
    \item{American Physical Society (2016 -- present)}
    \item{Optical Society of America (2020 -- present)}
\end{itemize}


\end{document}