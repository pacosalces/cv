%%%%%%%%%%%%%%%%%%%%%%%%%%%%%%%%%%%%%%%%%
% Medium Length Graduate Curriculum Vitae
% LaTeX Template
% Version 1.1 (9/12/12)
% Update to 1.1.1 (10/20/19), Francisco Salces--Carcoba
%
% This template has been downloaded from:
% http://www.LaTeXTemplates.com
%
% Original author:
% Rensselaer Polytechnic Institute (http://www.rpi.edu/dept/arc/training/latex/resumes/)
%
% Important note:
% This template requires the res.cls file to be in the same directory as the
% .tex file. The res.cls file provides the resume style used for structuring the
% document.
%
%%%%%%%%%%%%%%%%%%%%%%%%%%%%%%%%%%%%%%%%%

%----------------------------------------------------------------------------------------
%	PACKAGES AND OTHER DOCUMENT CONFIGURATIONS
%----------------------------------------------------------------------------------------

\documentclass[margin]{res} % Use the res.cls style, the font size can be changed to 11pt or 12pt here
\usepackage[backend=biber, maxbibnames=99, sorting=ydnt, doi=false, isbn=false]{biblatex}
\addbibresource{bibliography.bib}
\usepackage{hyperref}
\hypersetup{
    colorlinks = true,
    linkcolor = {red},
}
\usepackage{helvet} % Default font is the helvetica postscript font
%\usepackage{newcent} % To change the default font to the new century schoolbook postscript font uncomment this line and comment the one above

\setlength{\textwidth}{5.1in} % Text width of the document

\begin{document}

%----------------------------------------------------------------------------------------
%	NAME AND ADDRESS SECTION
%----------------------------------------------------------------------------------------

\moveleft.5\hoffset\centerline{\large\bf Francisco Salces-C\'arcoba}
\moveleft.5\hoffset\centerline{Country of citizenship: Mexico} 
\moveleft\hoffset\vbox{\hrule width\resumewidth height 1pt}\smallskip

\moveleft.5\hoffset\centerline{540 S Madison Ave Apt.2} 
\moveleft.5\hoffset\centerline{Pasadena, CA 91101}
\moveleft.5\hoffset\centerline{+1 (301) 222-3170}
\moveleft.5\hoffset\centerline{pacosalces@gmail.com}
\moveleft.5\hoffset\centerline{www.pacosalces.com}

%----------------------------------------------------------------------------------------
\nocite{*}
\begin{resume}

%----------------------------------------------------------------------------------------
%	EDUCATION SECTION
%----------------------------------------------------------------------------------------

\section{EDUCATION}

{\bf PhD,} Physics \\
Universiy of Maryland College Park, College Park, MD, May 2020 \\
Dissertation topic: Quantum simulation with ultracold bosonic gases\\
Title of dissertation: {\it Microscopy of elongated superfluids}

{\bf BSc,} Physics \\
Universidad Aut\'onoma de San Luis Potos\'i, San Luis Potos\'i, S.L.P., Mexico, 2013
 
%----------------------------------------------------------------------------------------
% SKILLS AND EXPERTISE SECTION
%----------------------------------------------------------------------------------------

\section{SKILLS AND EXPERTISE}

{\bf General:}
Experimental atomic, molecular, quantum and optical physics.\\
{\bf Hardware:}
Optical design in the VIS, NIR bands, high and ultra-high vacuum techniques to ${10^{-12} {\rm mbar}}$, miscellaneous electronic applications ranging DC to ${100 \, {\rm MHz}}$, and microwave sources at$~6.8\,{\rm GHz}$.\\
{\bf Software:}
Solidworks, Zeemax, Eagle, Python, \LaTeX, Labview, Matlab, VHDL. Personal public repositories at https://github.com/pacosalces

%----------------------------------------------------------------------------------------
%	PROFESSIONAL EXPERIENCE SECTION
%----------------------------------------------------------------------------------------
 
\section{RESEARCH TRACK}

{\sl Barish-Weiss Postdoctoral Scholar} \hfill 2020-present \\
{\bf Prototype of a cryogenic gravitational wave detector}\\
{\it California Institute of Technology, Pasadena, CA}
\begin{itemize} \itemsep -2pt
\item Develop $2\,{\rm \mu m}$ sources for next-generation gravitational wave detectors.
\end{itemize} 

{\sl Graduate research assistant} \hfill 2017-2020 \\
{\bf Holographic microscopy of ultracold ${\bf ^{87} {\rm \bf Rb}}$}\\
{\it Joint Quantum Institute (University of Maryland and NIST), Gaithersburg, MD}
\begin{itemize} \itemsep -2pt
\item Implement off-axis holographic microscope for ${^{87} {\rm Rb}}$ gases.
\item Measure and digitally filter aberrations using atomic shot noise.
\end{itemize} 

{\sl Graduate research assistant} \hfill 2016-2018 \\
{\bf New apparatus for quantum degenerate Bose gases}\\
{\it Joint Quantum Institute (University of Maryland and NIST), Gaithersburg, MD}
\begin{itemize} \itemsep -2pt
\item Design and assembly of ultra-high vacuum manifolds.
\item Design of magnetic quadrupole cold atom transport.
\item Design mounting structure and layout for the apparatus.
\end{itemize}

{\sl Graduate research assistant} \hfill 2015-2017 \\
{\bf Thermodynamics of one-dimensional Bose gases} \\
{\it Joint Quantum Institute (University of Maryland and NIST), Gaithersburg, MD}
\begin{itemize} \itemsep -2pt
\item Design, build and characterize high-aspect ratio, cross optical dipole trap.
\item Design, build and characterize compound microscope objective (${{\rm NA} = 0.31}$).
\item Calibrate optimal absorption imaging of dilute ${\sim 1 \,{\rm \mu m ^{-1}}}$ linear gases.
\item Benchmark {\it in-situ} density distributions with numerically exact model.
\end{itemize} 

{\sl Graduate research assistant} \hfill 2014-2015 \\
{\bf Digital control loop for magnetic field stabilization}\\
{\it Joint Quantum Institute (University of Maryland and NIST), College Park, MD}
\begin{itemize} \itemsep -2pt
\item Design and simulate 20-bit, FPGA-based feedback system for electromagnets.
\end{itemize} 

{\sl Undergraduate research assistant} \hfill 2011-2013 \\
{\bf Passive thermal stabilization of optical cavities}\\
{\it Laboratorio de Atomos Frios, Instituto de Fisica UASLP, S.L.P, Mexico}
\begin{itemize} \itemsep -2pt
\item Design and assembly of composite material spacers for confocal cavity.
\item Measure thermally driven frequency drift of the linear cavity. 
\item Model thermal transient with 4th-order Runge--Kutta finite-element code.
\end{itemize}
 
{\sl Undergraduate research intern} \hfill Summer 2012 \\
{\bf Soft X-ray calorimetry from ion electronic recapture}\\
{\it Oak Ridge National Lab, Oak Ridge, TN}
\begin{itemize} \itemsep -2pt
\item Operate ${{\rm keV}}$ molecular ion beam accelerator.
\item Operate high resolution X-ray cryogenic (${0.1 {\rm K}}$) calorimeter.
\end{itemize}

{\sl Undergraduate research assistant} \hfill 2009-2011 \\
{\bf Two-photon correlation functions $\bf g^{(2)}(\tau)$}\\
{\it Laboratorio de Atomos Frios, Instituto de Fisica UASLP, S.L.P, Mexico}
\begin{itemize} \itemsep -2pt
\item Automate measurements using two single photo detectors and oscilloscope.
\end{itemize}
 
\end{resume}
\section{PUBLICATIONS}
% we need heading=bibnumbered here to tell biblatex to use \section 
% not \section* (which will produce a spurious * with this class)
\printbibliography[heading=none]

\end{document}